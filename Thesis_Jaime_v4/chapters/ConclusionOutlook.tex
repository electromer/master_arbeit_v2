\chapter{Conclusion and Outlook}
\label{ch:conclusion}


In this thesis, a controller was developed to minimize the reflected mass for a 7-DOF LWR that is mounted on a linear axis. For this, several null space schemes are reviewed, on torque and velocity level. On torque level, projecting the gradient of the reflected mass into the null space shows poor performance close to maxima because the commanded torques are not able to overcome the friction of the rotational joints. An approach based on an attractive potential  solves this problem. For this potential, it is necessary to set a goal position. For obtaining the goal position, first global minimization is considered, so the goal position is selected to be the configuration with the lowest reflected mass in the null space.  For a given trajectory, the computation of the minimum in each time step may be done offline. However, the global minimum is prone to discontinuities, so a local minimum is preferred for the attractive potential. \\
For the search of the local minimum in reflected mass, two fundamental optimization techniques are considered, namely Trust Region and Line Search techniques. The Trust Region techniques require an analytical form of the reflected mass constrained to the null space motion. Since it is not possible to obtain an analytical form, the Line Search techniques are preferred in this work. Therefore, a null space velocity that minimizes the reflected mass is required. The first Line Search technique considered is the Gradient Descent. This one is implemented using the Projected Gradient, and the Reduced Gradient method, where the gradient is eight- and two-dimensional respectively. The Reduced Gradient is  chosen for being more computationally efficient. This method also allows to weight the minimization w.r.t. the redundant joints. Being possible to choose the joints used for the minimization. By exploiting this characteristic, a joint limit avoidance strategy is developed.  The Reduced Gradient is obtained analytically and numerically. Inherent to the gradient descent method, oscillations close to the minima appear when the problem is ill-conditioned (high condition number). Decreasing the step size causes slow motion, so other line search techniques are implemented in this thesis in order to decrease oscillations and increase convergence rate. Three Conjugate Gradient Descent methods are tested, showing worse convergence than the original Gradient Descent. Two Quasi-Newton methods are tested as well, showing not a significant improvement in the convergence rate compared to the Gradient Descent. Therefore, the Gradient Descent is chosen for simplicity. %Still showing oscillations, an algorithm is developed in which several iterations are computed in order to detect and decrease these oscillations. 
Still showing oscillations, a  non-normalized joint velocity is chosen. This bounded velocity   shows good dynamics in the minimization and ensures no oscillations close to minima.

An algorithm based on the Reduced  Gradient is implemented in a real-time capable controller. The controller is tuned for both, the analytical and the numerical approach. The addition of damping is necessary to decrease the overshooting in the joints, especially in the linear axis. The analytical approach is  computationally more efficient. The numerical approach is more robust to parameters variation.
Finally the controller is implemented in combination with the SMU. The SMU limits the velocity to a biomechanically safe value provided the mass and the curvature in the direction of movement. While the SMU ensures safe motion, the controller developed in this thesis minimizes the reflected mass, allowing the end-effector to reach higher velocities. 



%\section{Future work}
%\label{sec:future_work}
%
%Prove analytically that the integral of kernel of the Jacobian of the LWR (\ref{eq:dq_ns}) has no closed form. An attempt has been done in this work finding the most simple terms of this kernel and searching for a closed form of their integrals. Although software like Mathematica is not able to obtain these closed forms directly, in this thesis it is proved that some of them have closed form. Further analysis of the remaining terms is left to future work, since proving that one of this terms has no closed form would be enough to prove that the integral of the kernel has not closed form either. 
%\subsection{Generalization to n-DOF}
%\label{subsec:extension_ndof}
%





%\section{Future work}
%-more efficient ways to compute these matrices and their inverses to allow higher nmax. Specially in the numerical approach

