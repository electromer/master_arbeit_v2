\chapter{Experiments}
\label{ch:experiments}



The experiments carried out can be divided into experiments where the end-effector (EE) is not moving  and others where the EE is following a trajectory. 
For the static case EE (Section \ref{sec:static_case}), the three Cartesian directions are chosen to minimize the reflected mass . The goal of the tuning is to find a common set of parameters for the minimization along all three directions, including the effect of external forces. If accomplished, it would not be necessary to change the parameters of the controller when the robot changes its direction of motion.\\
%
For the dynamic cases (Section \ref{sec:dynamic_case}) the  experiment is the motion of the robot along the linear axis. This experiment is performed at low and high speeds, 0.2 and 1 $\mathrm{m/s}$ respectively. 
The aim is to find a set of parameters that performs well in all cases, and if possible, a common set of parameters for both approaches, numerical and analytical.
Finally, both approaches are tested integrated together with the Safe Motion Unit (SMU). 

Before starting with the experiments, it is worth summarizing here the different parameters that will play a role. 
The RG is implemented on the final controller, so the first parameters to tune depend on which approach is used: Analytical of numerical. If using the analytical approach (see \ref{eq:qdot_analytical}), the first two parameters to tune are the scaling factor $\mathrm{\gamma}$ and the desired reflected mass $\mathrm{m_{des}}$. The numerical approach does not add any parameters. Once the null space velocity is obtained (see line \ref{2D_alg:line:select} in Algorithm \ref{alg:2D_alg}), it is integrated using Euler method with step size $\mathrm{dt}$. This algorithm runs maximum $\mathrm{n_{max}}$ iterations before commanding a goal position. The commanded torque is obtained in (\ref{eq:torque_pot_intro}) multiplying by the gain $\mathrm{K_p}$. Besides the proportional action, a derivative action is added to the controller in order to decrease overshooting.



\section{Static Case} 
\label{sec:static_case}







Here, the three main Cartesian directions (x,y,z) are used separately as direction to minimize the reflected mass. These directions are in the world frame. In Fig. \ref{fig:reflected_mass_cart_directions}  one can see that the reflected mass in the x- and z-direction depends mainly on the elbow ($\mathrm{q_4}$). In the y-direction it depends on both, which is to be expected since the linear axis ($\mathrm{q_1}$) is orientated along the y-direction. Looking at the axes of the three grids in Fig. \ref{fig:reflected_mass_cart_directions} one can notice how the linear axis' range is approximately one, while for the elbow it is around 10. In these figures no joint limits are considered. But even considering them, there is a significant difference between the scales of both joints. Therefore, non-weighted  optimization ($\mathrm{WF = 0.5}$) leads to a greater displacement of the linear axis in the grid. \\
%
\begin{figure}[htp!]
	\centering	
	\subfigure[]{\label{fig:reflected_mass_x}}{\includegraphics[height=0.35\textwidth]{images/reflected_mass_x-crop.pdf}} 	\subfigure[]{\label{fig:reflected_mass_y}}{\includegraphics[height=0.35\textwidth]{images/reflected_mass_y-crop.pdf}} 	
	\subfigure[]{\label{fig:reflected_mass_z}}{\includegraphics[height=0.35\textwidth]{images/reflected_mass_z-crop.pdf}} 	
	\caption{Plot of the reflected mass in x, y, and z direction obtained in \ref{sec:Fabianstuff}}
	\label{fig:reflected_mass_cart_directions}
\end{figure}
%
%
Minimizing in the y-direction (see Fig. \ref{fig:y_dir_minim_SC}) the robot tends to align the EE with the the linear axis.
%\begin{figure}[htb]
%	\centerline{
%		\includegraphics[width=1.0\textwidth]{images/y_dir_minim_SC.eps}}
%	\caption{Trajectory described by the robot when minimizing in y-direction. The initial configuration is depicted in blue and the final one in red {}
%	\label{fig:y_dir_minim_SC}
%\end{figure}
\begin{figure}[htp!]
	\centering	
	\subfigure[]{\label{fig:y_dir_minim_SC}}{\includegraphics[height=0.4\textwidth]{images/y_dir_minim_SC.eps}} 	\subfigure[]{\label{fig:x_dir_minim_SC}}{\includegraphics[height=0.43\textwidth]{images/x_dir_minim_SC.eps}} 	
	\subfigure[]{\label{fig:z_dir_minim_SC}}{\includegraphics[height=0.4\textwidth]{images/z_dir_minim_SC.eps}} 	
	\caption{Trajectory described by the robot when minimizing in the y-,x- and z-direction. The initial configuration is depicted in blue and the final one in red. }
	\label{fig:dir_minim_SC}
\end{figure}
%
%
\\ Secondly, the x-direction is tested. In this case the robot moves from elbow-up to elbow-down as it can be seen in Fig. \ref{fig:x_dir_minim_SC}.  With the numerical approach, good behavior is achieved for a wide range of parameters. And these parameters  work fine for the y-direction as well. On the other hand, the same parameters cannot be used when using the analytical approach in both directions. The scaling factor $\mathrm{\gamma}$ has to be decreased w.r.t. the one used in the y-direction. This can be explained by looking at the higher initial mass in the second tested direction. In  Fig. \ref{fig:reflected_mass_cart_directions}  one can see that on the x-direction the mass will usually be between 10-18 kg while along the y-direction it will usually be between 6-14 kg. Meaning that the scaling of the RG is almost twice as high minimizing in the x-direction w.r.t the y-direction.  
\\
%
The z-direction is a special case because the null space grid (see Fig. \ref{fig:reflected_mass_z}) shows an contour line for the local minima, existing two equal global minima in both extremes of the linear axis. This causes the linear axis to separate from the EE. Therefore singularities, whose treatment is out of the scope of this work, may appear. %The joint limit avoidance implemented on the algorithm will stop the minimization once the conservative joint limits are reached but the inertia on the joints can make them reach the area between the conservative and the real limits. For this a better approach is to use a repulsive force for the joints that are beyond their limits.{\color{red} this I have not implemented yet}
The trajectory described by the robot in the z-direction can be seen in Fig. \ref{fig:z_dir_minim_SC}




In Table \ref{table:table_SC}, the values for the mentioned experiments are depicted. These values show a good behavior in the real robot. The main difference between the experiments on the real robot and the real-time simulations, is that the gains have to be higher than in the simulation to overcome the friction in the real robot. Specially the gains for the rotational joints need to be increased. As it happened in Section \ref{sec:Gradientbasedminimization}, the rotational joints have a high friction, causing the real robot not to move for small torques. Although we are on velocity level, the number of iterations for the real robot has to be increased w.r.t the real time simulation. 
 In Table \ref{table:table_SC} one can see how the number of iterations required for the numerical approach is much higher than for the analytical approach, having both similar values for the step size and $\mathrm{K_p}$. Furthermore, the analytical RG needs to be again  reduced, scaling it by $\mathrm{\gamma}$. This shows again the high sensitivity of the parameters for the analytical RG, but also shows an advantage of this method: The lower number of iterations makes this approach more computationally efficient. The sensitivity of the analytical approach comes from the non-normalization of the velocity. Using a normalized velocity, the controller shows poor dynamics when it is following a trajectory. To improve the dynamics one may increase $\mathrm{K_p}$ but this cause oscillations close to the minimum. Non-normalized velocities are used to get rid of the oscillations, but the velocities obtained from the analytical approach are much higher than the ones from the numerical approach. This explains the need to decrease the velocity with so low values of $\mathrm{\gamma}$, and a scaling gain that decreases when it gets close to the minimum.
As mentioned in Section \ref{sec:analytical},  $\mathrm{\dot{m}_u}$ is nothing but a variable scaling factor that improves the dynamics of the robot. For the analytical approach this factor contributes decreasing the joint velocities close to the minimum, decreasing the overshooting.
For the numerical approach the behavior is good enough with a constant scaling gain, so for simplicity of tuning a variable gain is not implemented.  \\
As the mass depends on the  configuration of the robot one cannot know a priori what value should be selected for $\mathrm{m_{des}}$. A value over the current mass will lead to a increase of the mass w.r.t time. 
For the sake of safety a value of $\mathrm{m_{des}=0}$ kg is set by default. It is worth mentioning that if a minimum threshold for the mass is known then it is desirable to set this value as the desired mass because this leads to smoother behavior when the minimum is reached.




For the damping of the joints three different designs are implemented. First a simple diagonal matrix. Besides this one, a factorization design and a double diagonalization design \cite{alin_damping} are  implemented in this thesis. The tests to tune the damping were  done with a static EE position, while exerting external forces on the robot. The factorization and double diagonalization designs showed very similar results, and in both cases better than the simple diagonal matrix. But it is necessary to use different damping for the linear axis and for the rotational joints. The Cartesian stiffness, and damping controller  already add enough damping to the rotational joints, therefore small damping needs to be added to these joints. On the other hand, the linear axis shows high overshooting, specially when minimizing in the y-direction. Therefore, the  damping ratio ($\mathrm{\zeta}$) of the linear axis is chosen to be one. For the rotational joints $\mathrm{\zeta}$ around 0.4 is preferred.
This damping also helps to decrease the oscillations inherent to the GD method.







\begin{table}[]
	\centering
	\caption{Parameters for the static case in the x-, y- and z-direction}
	\label{table:table_SC}
	\begin{tabular}{@{}|c|c|c|c|c|c|c|@{}}
		\toprule
		\textbf{Direction}          & \textbf{Approach}   & \textbf{Step size} & \textbf{Number iterations} &\textbf{$\mathrm{\zeta}$} & \textbf{$\mathrm{\gamma}$} & \textbf{$\mathrm{K_p}$}\\ \midrule
		\multirow{2}{*}{\textbf{X}} & \textbf{Numerical}  & 5e-4 to 5e-3       & 10-30                      & \multirow{2}{*}{0.4/1} & -     & \multirow{2}{*}{10-100} \\ \cmidrule(lr){2-4} \cmidrule(lr){6-6}
		& \textbf{Analytical} & 5e-3               & 2-4                        &                        & 1e-3  &                         \\ \midrule
		\multirow{2}{*}{\textbf{Y}} & \textbf{Numerical}  & 5e-4 to 5e-3       & 10-30                      & \multirow{2}{*}{0.4/1} & -     & \multirow{2}{*}{10-100} \\ \cmidrule(lr){2-4} \cmidrule(lr){6-6}
		& \textbf{Analytical} & 5e-3               & 2-4                        &                        & 1e-2  &                         \\ \midrule
		\multirow{2}{*}{\textbf{Z}} & \textbf{Numerical}  & 5e-4 to 5e-3       & 10-30                      & \multirow{2}{*}{0.4/1} & -     & \multirow{2}{*}{10-100} \\ \cmidrule(lr){2-4} \cmidrule(lr){6-6}
		& \textbf{Analytical} & 5e-3               & 2-4                        &                        & 1e-2  &                         \\ \bottomrule
	\end{tabular}
\end{table}







\section{Dynamic Case}
\label{sec:dynamic_case}


The  experiment carried out is the motion of the EE along the linear axis, which shows the full potential of minimizing with a LWR mounted on a linear axis.
First of all different proportional gains (10, 20, 30 and 100) are tested. For each of these gains (common for all joints) different step sizes are tested. Starting from the optimal step size of 5e-2 s a ten times bigger and ten times smaller step size are tested. For all of these combinations of gains and step size, five different numbers of iteration are used: 4, 10, 20, 50 and 100.
The motion consists of moving the EE 60 cm  in the y-direction with Cartesian control.
A larger distance was not chosen  because the original robot behavior tends to stretch the elbow completely,  leading to singularities. 
%As explained in the previous section the minimization in z-direction caused the robotic arm to stretch too much, reaching singularities. Although setting higher conservative joints limits will avoid singularities the minimization is then very small and to show better the behavior of the controller we will limit the experiments to the case in y-direction. 
%TODO:try to do minimization along x and z dynamic just to show that the previous parameters work} {\color{red} delete this paragraph  if I have time to do more experiments and put them here
 Firstly the previously obtained  parameters from Table 	\ref{table:table_SC} are tested at high speeds, and later at lower speeds. At high speeds a maximum velocity of 1 $\mathrm{m/s}$ is achieved during the motion with a maximum acceleration of 10 $\mathrm{m/s^{2}}$, and at lower speeds the maximum velocity is 0.2 $\mathrm{m/s}$, with the same maximum acceleration.





\subsection{Numerical Controller}




The minimized reflected mass is the first result that has to be compared. For this, the parameters that cause bad minimization are discarded. This happens for all the cases with step size 5e-1 s, and in the cases with step size  5e-2 s and low values of the other parameters. 5e-3 s gave in general better results so we will carry one with this value, and not lower because 5e-4 s causes a  slow motion.
\begin{figure}[!htb]
	\centerline{
		\includegraphics[width=0.8\textwidth]{images/kp100dt5e-3_v2.eps}}
	\caption{Motion of the linear axis when moving the TCP 60 cm along the y-direction for different number of iterations. The numerical approach is used with a maximum velocity of 1 $\mathrm{m/s}$..}
	\label{fig:kp100dt5e-3}
\end{figure}

\begin{figure}[!htb]
	\centerline{
		\includegraphics[width=0.8\textwidth]{images/mass_kp100dt5e-3_v2.eps}}
	\caption{Error in the reflected mass when moving the end-effector 60 cm along the y-direction for different number of iterations. The numerical approach is used with a maximum velocity of 1 $\mathrm{m/s}$.}
	\label{fig:mass_kp100dt5e-3}
\end{figure}

Another criterion is the computational efficiency, and here the number of iterations plays the biggest role. The numerical approach is more computationally expensive than the analytical and care must be taken when choosing $\mathrm{n_{max}}$.  For the numerical approach with $\mathrm{n_{max} > 30}$, the frequency of the algorithm has to be decreased from 1kHz to 100Hz.  With the algorithm running at 1 kHz, the load in one core of the CPU is up to 60\% for $\mathrm{n_{max} = 30}$. While for a low number of iterations ($\mathrm{n_{max} \approx 4} $) this load is around 5\%.  At 100Hz and $\mathrm{n_{max} = 100}$ the load of the CPU is between 50 and 70\%.  Furthermore, for  $\mathrm{n_{max} \ge 10}$,  $\mathrm{K_p=100}$ and step size 5e-3 s,  there is small overshooting (see Fig. \ref{fig:kp100dt5e-3}). This occurs for both approaches in the linear axis, even for $\mathrm{\zeta = 1}$. In the same Fig. \ref{fig:kp100dt5e-3} one can see how when the number of iterations is too low the linear axis ends farther from the EE (which in this case is in 0.6 m). In the Fig. \ref{fig:mass_kp100dt5e-3} this corresponds to a higher mass at the end of the motion. 
For $\mathrm{K_p} \approx 10 - 30$, the results are very similar as long as we limit the output torques from the controller to $\pm$20 Nm. With these low gains the minimization is only slightly worse than with higher gains. Meaning that the numerical approach is more robust to the tuning of the proportional gains, and also meaning that gains between 30 and 100 lead to good behavior.
Of course this robustness would be lower if starting very far from the minimum. With  high $\mathrm{n_{max}}$, and high $\mathrm{K_p}$, the robot would have high initial torques that could cause higher overshooting than the one showed in Fig. \ref{fig:kp100dt5e-3}. For this reason gains around 30 are preferred. 

To sum up the parameters for the dynamic case with the numerical approach are:
 \begin{itemize}  
 	\item \textbf{Step size} : 5e-3 s
 	\item \textbf{Kp}: 10-30
    \item \textbf{Number of iterations}: 10-30
  \end{itemize} 
 
As already mentioned, the velocity obtained in the algorithm is not normalized before  integration, but it is bounded instead. This is done because a normalized velocity shows a poor minimization for the dynamic cases when using both approaches. These may be solved by increasing the proportional gains significantly in the controller. But then  oscillations appear close to the minimum because the error is  amplified. To keep these gains low and still achieve good dynamics the only option is to increase the step size and/or the number of iterations between each sent command. But these two parameters should be kept low as already explained. \\
Non-normalized joint velocities are chosen instead. These velocities are higher far from the minimum which means better response when the minimum is changing, like it occurs when the EE moves, and small close to the minimum so no oscillations appear. However, this  velocity still has to be limited properly to ensure that the robot will move inside the null space grid.  In \cite{fabianthesis} a normalized velocity with a step size ($\mathrm{DT=0.5}$)was used to test the integration error when creating the null space grid for a specific configuration. 
In our case this step size ($\mathrm{DT}$) to create the grid can be approximated to the step size ($\mathrm{dt}$) used in the algorithm, times the non-normalized velocity. Using a normalized velocity, in the worst case scenario there is a joint which position increases $\mathrm{DT}$ after each step. So without normalizing the velocity we have to ensure that the joints' positions have a maximum increase of $\mathrm{DT}$.
The step size in the algorithm ($\mathrm{dt}$) is 5e-3 s, which means that to stay under $\mathrm{DT=0.5}$ the velocity has to be less than 100 ( $\mathrm{m/s}$ for the linear axis and $\mathrm{rad/s}$ for the rotational joints). In the experiments carried out this limit is never reached which concludes that the non-normalized velocity is an adequate choice. 


\subsection{Analytical Controller}


Now the same experiment is carried out, but using the analytical approach.
Also here the proportional gains 10, 20, 30 and 100 are tested.  Regarding the number of iterations, it is kept to the minimum ($\mathrm{n_{max} \approx 2-4}$) because, as in the static cases, this approach shows to be very sensitive to changes of this parameter.
The desired mass is always zero as explained before.  Equivalent results to the ones from  Fig. \ref{fig:mass_kp100dt5e-3} can be seen in Fig. \ref{fig:analyt_cm3} for the analytical case. Starting close to the minimum $\mathrm{\gamma = 1e-2}$ shows good minimization. But when doing a similar motion starting a bit farther from the minimum (see see Fig. \ref{fig:analyt_cm2}) the effect of $\mathrm{\gamma}$ is different. In this second motion the minimum is not found and during the trajectory the peak is even higher than without controller what is something to avoid. This is because it starts farther from the minimum, with higher RG, which causes a fast motion at the beginning. Some of the joint limits are reached, causing the minimization to stop. Evolving from this new configuration without minimization leads to a higher mass. In this tests the joint limit avoidance strategy is not implemented. But this is just to show again the high sensitivity of the analytical approach to the changes in $\mathrm{\gamma}$.
Therefore a lower $\mathrm{\gamma}$ is chosen, although the local minimum is not found. 
For lower proportional gains the effect of $\mathrm{\gamma}$ is not so high so gains around 30 are preferred.
For proportional gains of 30 the table \ref{table:dynamic_opt_values} shows the most favorable values of the parameters for this experiment. \\
%
%
\begin{figure}[!htp!]
	\centering	
	\subfigure[]{\label{fig:analyt_cm3}}{\includegraphics[height=0.37\textwidth]{images/analyt_cm3_v2.eps}} 	\subfigure[]{\label{fig:analyt_cm2}}{\includegraphics[height=0.37\textwidth]{images/analyt_cm2_v2.eps}} 	 	
	\caption{Difference between the desired reflected mass and the current one for the analytical approach with a maximum velocity of 1 $\mathrm{m/s}$.   Fig. \ref{fig:analyt_cm3} starting in the minimum and
 	  Fig. \ref{fig:analyt_cm2} starting farther from the minimum. Both motions are for $\mathrm{K_p = 100, dt=5e-3}$.}
	\label{fig:analyt_cm2and3}
\end{figure}
%
As in the numerical case we still have to prove that the velocities reached the joint configurations are inside the null space grid. For this we have a step size scaled by $\mathrm{\gamma}$ which in the ends is equivalent to a step size ($\mathrm{dt}$) of 500ns. In order to ensure that the limit ($\mathrm{DT=0.5}$) is not reached, the velocity has to be bounded by 1e6 which is a value way higher than the ones reached in practice.


The parameters found for the static case work well at low speeds in both approaches. However,  the parameter at high speed in the analytical approach do not show good performance neither at low speeds, nor with static EE.

\begin{table}[]
	\centering
	\caption{Dynamic motion along y}
	\label{table:dynamic_opt_values}
	\begin{tabular}{@{}|c|c|c|c|c|c|@{}}
		\toprule
		\textbf{Approach}   & \textbf{Step size} & \textbf{Number iterations} & \textbf{$\mathrm{\zeta}$} & \textbf{$\mathrm{\gamma}$} & \textbf{$\mathrm{K_p}$}\\  \midrule
		\textbf{Numerical}  & 5e-4 to 5e-3       & 10-30                      & \multirow{2}{*}{0.4/1} & -     & \multirow{2}{*}{$\sim$30} \\ \cmidrule(r){1-3} \cmidrule(lr){5-5}
		\textbf{Analytical} & 5e-3               & 2-4                        &                        & 1e-3  &                           \\ \bottomrule
	\end{tabular}
\end{table}


\section{Integration with SMU}

The last section of this chapter regards the integration of the controller with the Safety Motion Unit (SMU). Depending on the reflected inertia, velocity, and geometry at possible impact locations, the SMU generates truly safe velocity bounds considering the dynamic properties of the manipulator and human injury. The minimization controller is implemented together with the SMU to prove that a safe motion and better performance are achieved.
For the experiment we first describe the EE to test the performance of the controller.


\subsection{End-effector description}
In \cite{sammi_paper} an EE composed of the primitives from the drop-testing experiments from the same paper was considered. The same EE will be used here with the controller.\\
%
This EE, illustrated in  Fig. \ref{fig:EE_description}, is composed of four Points Of Interest (POI) attached to four different geometries: POI1 is attached to a small sphere, POI2, POI3 to the edge, and POI4 to the large sphere. 
In the experiment, the motion will be similar to the one carried out in Section \ref{sec:dynamic_case}, where the robot is moving along the linear axis. In this way both safety curves of POI1 and POI4 will be considered not considering here POI2 and POI3.
In this experiment the robot has a minimum at approx 5.4 kg. For this minimum the safety curves created in  \cite{sammi_paper} have a maximum permissible velocity over 1 $\mathrm{m/s}$. These velocities in the real robot are not implementable. Therefore, the safety curves are modified to have lower maximum velocities in order to implement the controller in the real robot. Therefore new safety curves are created at lower speeds for demonstration purposes.





\begin{figure}[t!]
	\centerline{
		\includegraphics[width=0.5\textwidth]{images/EE_description.png}}
	\caption{POIs associated to the impactor primitives in the end-effector \cite{sammi_paper}}
	\label{fig:EE_description}
\end{figure}



\begin{figure}[t!]
	\centerline{
		\includegraphics[width=0.5\textwidth]{images/EE_in_real_robot.png}}
	\caption{End-effector coordinates expressed in the world base frame for the experiment. The $\mathrm{{}^{0}_{}y}$ axis points towards the small sphere primitive. \cite{sammi_paper}}
	\label{fig:EE_in_real_robot}
\end{figure}





\subsection{Results}

For the first motion, the robot is commanded to move the EE, pure translation, 60 cm in the opposite direction of $\mathrm{{}^{0}_{}y}$  (y-axis of the world base frame). When this position is reached the robot is commanded to go in the opposite direction to the original position.
The whole motion is first done without controller and velocity limitation from the SMU. In the second experiment, the SMU is activated and the velocity of the robot is possibly limited to ensure a safe motion. During the first motion, the small sphere (POI1) defines the safety curve. Coming back to the original position the bigger sphere (POI4) defines the new safety curve and the bounds of the velocity. Without SMU, the maximum velocity of the robot is limited to 0.5 $\mathrm{m/s}$ for a better comparison of the results. \\
In  Fig. \ref{fig:SmuExp_ErrMass_vmax05} one can see the error in the reflected masses for these approaches. Meaning by error, the difference between the current reflected mass and the mass in the local minimum. At the beginning of the first motion (Fig. \ref{fig:SmuExp_ErrMass_CM2_POI1_vmax05}) the reflected mass with all approaches is the same since they start in the same configuration, but at the end of this motion the reflected mass is much lower using the controller. The masses at the end of the first motion, correspond to the masses at the beginning of the second motion (Fig. \ref{fig:SmuExp_ErrMass_CM3_POI4_vmax05}). The numerical and analytical approach show very similar performance, what is to be expected from the experiments run in Section \ref{sec:dynamic_case}.
%
%
%
\begin{figure}[htb!]
	\centering	
	\subfigure[]{\label{fig:SmuExp_ErrMass_CM2_POI1_vmax05}}{\includegraphics[height=0.55\textwidth]{images/SMU_test_3/SmuExp_ErrMass_CM2_POI1_vmax05_ubuntu.eps}} 	\subfigure[]{\label{fig:SmuExp_ErrMass_CM3_POI4_vmax05}}{\includegraphics[height=0.55\textwidth]{images/SMU_test_3/SmuExp_ErrMass_CM3_POI4_vmax05_ubuntu.eps}} 	
	\caption{Reflected masses over the local minimum using the controller integrated with the SMU. The curves without controller and  SMU limitation are also depicted. The first motion is moving along the linear axis 60 cm (Fig. \ref{fig:SmuExp_ErrMass_CM2_POI1_vmax05}), and the second one is going back to the original position (Fig. \ref{fig:SmuExp_ErrMass_CM2_POI1_vmax05}).}
	\label{fig:SmuExp_ErrMass_vmax05}
\end{figure}
%
%
In Fig. \ref{fig:SmuExp_Vel_vmax05} the absolute value for the velocity in the EE is depicted. The controller minimizes the reflected mass, and the lower the mass the higher is the velocity limit.  In Fig. \ref{fig:SmuExp_Vel_CM2_POI1_vmax05} one can see clearly how the velocities reached with the controller are higher than without the controller. This causes the EE to reach the final position in less time, which corresponds to the moment when the velocities drop to zero. 
In these figures one can also see how at the beginning, and at the end, of the motion the velocities show small peaks. Since we are plotting the absolute velocity, this is a sum of the three Cartesian velocities in the EE. This peak is higher in the x-direction that in the y- and z-, but still appears in the three directions. These peaks are due to the high acceleration of the experiment and using lower accelerations these peaks decrease.  \\ 
In Fig. \ref{fig:SmuExp_SafetyCurve_vmax05} the reflected masses are depicted, together with the corresponding velocities for each instant of the trajectory. Here one can see the effect of the SMU limiting the velocity, and how the controller reduces the mass allowing a faster motion. The safety curves are also depicted.
%
The points where the velocities exceed slightly the allowed maximum correspond to the mentioned peaks from Fig. \ref{fig:SmuExp_Vel_vmax05}. The SMU usually starts limiting the velocity only when this reaches the safety curve.  However, this causes a big overshooting. These velocities show also oscillations close to the safety curve, but these oscillations may be decreased using a filter. To ensure a safe motion  during the overshooting, one may set a conservative safety curve, with lower limits than the original one. Another option is to start limiting the velocity earlier. In the integration of the SMU in the Cartesian interpolator, this can be understood as using a smoother signal, instead of a step, as input to the interpolator.    Therefore, the limitation of the velocity is smoother and no overshooting is shown. In Fig. \ref{fig:SmuExp_SafetyCurve_vmax05} this earlier limitation appears when the blue curves start to separate from the black one. This approach is preferred for this experiment since the overshooting is much lower than if the velocities start being limited when the safety curve is reached. 

\begin{figure}[htp!]
	\centering	
	\subfigure[]{\label{fig:SmuExp_Vel_CM2_POI1_vmax05}}{\includegraphics[height=0.55\textwidth]{images/SMU_test_3/SmuExp_Vel_CM2_POI1_vmax05_ubuntu.eps}} 	\subfigure[]{\label{fig:SmuExp_Vel_CM3_POI4_vmax05}}{\includegraphics[height=0.55\textwidth]{images/SMU_test_3/SmuExp_Vel_CM3_POI4_vmax05_ubuntu.eps}} 	
		\caption{Absolute velocities in the EE moving along the linear axis 60 cm (Fig. \ref{fig:SmuExp_Vel_CM2_POI1_vmax05}), and moving back to the original position (Fig. \ref{fig:SmuExp_Vel_CM3_POI4_vmax05}). First is shown the standard motion of the robot, then with velocity limitation from the SMU and then with both the numerical and analytical approaches integrated together with the SMU.}
	\label{fig:SmuExp_Vel_vmax05}
\end{figure}




\begin{figure}[htp!]
	\centering	
	\subfigure[]{\label{fig:SmuExp_SafetyCurve_CM2_POI1_vmax05}}{\includegraphics[height=0.55\textwidth]{images/SMU_test_3/SmuExp_SafetyCurve_CM2_POI1_vmax05_thin_ubuntu.eps}} 	\subfigure[]{\label{fig:SmuExp_SafetyCurve_CM3_POI4_vmax05}}{\includegraphics[height=0.55\textwidth]{images/SMU_test_3/SmuExp_SafetyCurve_CM3_POI4_vmax05_thin_ubuntu.eps}} 	
	\caption{Velocities of the EE and reflected masses the end-effector, when the robot is moving along the linear axis 60 cm (Fig. \ref{fig:SmuExp_SafetyCurve_CM2_POI1_vmax05}), and moving back to the original position (Fig. \ref{fig:SmuExp_SafetyCurve_CM3_POI4_vmax05}). The safety curves for each motion are also depicted.}  
	\label{fig:SmuExp_SafetyCurve_vmax05}
\end{figure}


\section{Discussion}



For the numerical approach the parameters found at high speeds (1 $\mathrm{m/s}$) work fine at lower speeds. At low speeds (0.2 $\mathrm{m/s}$) a higher number of iterations ($\mathrm{n_{max} \approx 20}$) and high gains ($\mathrm{K_p \approx 100}$) would give a bit better minimization (up to 20\% lower reflected mass). But still with these parameters the optimization is fine. \\
%
 For the analytical approach, the parameters found at high speeds do not perform well at low speeds. At low speeds  $\mathrm{\gamma}$ has to be at least 1e-2. At high speeds such high $\mathrm{\gamma}$ causes very high torques, and some joints tend to reach their limit. The high sensitivity of the analytical approach is due to its higher non-normalized velocities. An alternative approach would be setting a higher $\mathrm{n_{max}}$ with normalized velocities. Normalized velocities would make the tuning more robust since the velocity bounds are smaller, and the high number of iterations would ensure good dynamics. Using low $\mathrm{K_p}$  no oscillations would appear close to the minimum. And the computational efficiency would still allow real time capabilities, since this approach is not so computationally expensive as the numerical one.   
 However, for this alternative to be possible, one must ensure that there are not local maxima between the current configuration and the local minimum (see Fig. \ref{fig:bad_case_3d}).\\
%
At low speeds and in the static case the minimization is similar with both approaches. In the tests carried out with the SMU both approaches showed good performance. The analytical one reached slightly lower values of the mass. 


Regarding the damping, the rotational joints could easily be damped to reach critically damped behavior and the same values showed good performance in the dynamic cases at low speeds. The linear axis was more prone to overshoot and in the dynamic cases the factorization design and the double diagonalization design showed similar and better performance than the simple diagonal matrix even for high damping (D=10) of the linear axis. 
The Cartesian controller ads already enough damping to the rotational joints. And since they also have high friction, damping ratios in the range 0.2-0.3 are chosen for these joints. The linear axis needs higher damping, since it has an admittance interface with position controller. To avoid overshooting a damping ratio of 1 is finally chosen for both damping designs from \cite{alin_damping}.



